\documentclass{article}
\usepackage[utf8]{inputenc}
\usepackage[brazil]{babel}
\usepackage{amsmath} % Ensure amsmath is loaded for equation* environment
\usepackage{amssymb} % For \textbullet and \textcirc
\title{Kori}
\author{Pedro Henrique T. A.}
\date{Outubro 2025}

\begin{document}
\maketitle

\section{Introdução}
O tema deste trabalho é a construção de um sistema de recomendação de animes, que tem como objetivo auxiliar fãs e interessados a descobrirem novos títulos a partir de suas preferências. O sistema utiliza técnicas de processamento de linguagem natural para analisar descrições de animes e calcular a similaridade entre eles, recomendando aqueles que mais se assemelham ao escolhido pelo usuário.

Para isso, é empregado o algoritmo TF-IDF (Term Frequency-Inverse Document Frequency), que transforma as sinopses dos animes em vetores numéricos, permitindo comparar o conteúdo de forma matemática. A similaridade entre os animes é então calculada utilizando o cosseno entre esses vetores, identificando quais títulos possuem maior proximidade temática e textual.

Todo o código foi escrito em Python, utilizando bibliotecas como Pandas, Colorama, NumPy e NLTK. Os objetivos principais são:
\begin{itemize}
    \item Fornecer recomendações de animes similares a partir de um anime escolhido pelo usuário.
    \item Aplicar conhecimentos da disciplina de Álgebra Linear do curso de Ciência de Dados da FATEC - Rubens Lara, ministrados pelo professor Alexandre Garcia de Oliveira, demonstrando aplicações reais de algoritmos de similaridade e espaços vetoriais.
\end{itemize}

\section{Funcionamento}
\subsection*{Etapas do Sistema}
O funcionamento do sistema segue as etapas abaixo:
\begin{enumerate}
    \item O usuário pesquisa pelo nome do anime desejado e seleciona um título dentre os resultados.
    \item São gerados tokens (palavras relevantes) tanto da sinopse do anime escolhido quanto dos resumos dos episódios, enriquecendo a representação textual desse anime.
    \item Para cada anime do dataset, é criado um documento individual, também baseado em sua sinopse e informações relevantes.
    \item O sistema utiliza o TF-IDF para transformar todos esses textos em vetores numéricos.
    \item Por fim, usando a similaridade dos cossenos, compara o vetor do anime escolhido com os vetores dos demais animes do dataset (query), identificando aqueles com maior similaridade para recomendar ao usuário.
\end{enumerate}
\subsection*{Análise dos Ângulos entre Vetores}
No sistema de recomendação, cada anime é representado por um vetor numérico gerado a partir de informações relevantes usando o algoritmo TF-IDF. Para medir a similaridade entre dois animes, calcula-se o ângulo entre seus vetores no espaço multidimensional.
O cosseno do ângulo ($\cos(\theta)$) entre dois vetores $\vec{A}$ e $\vec{B}$ é dado por:
\[
    \cos(\theta) = \frac{\vec{A} \cdot \vec{B}}{\|\vec{A}\| \; \|\vec{B}\|}
\]
\begin{itemize}
    \item Se o ângulo é pequeno ($\cos(\theta)$ próximo de 1), os vetores são próximos e os animes são muito similares.
    \item Se o ângulo é grande ($\cos(\theta)$ próximo de 0), os vetores são distantes e os animes são pouco similares.
\end{itemize}
Assim, o sistema recomenda os animes cujos vetores formam os menores ângulos com o vetor do anime escolhido pelo usuário, ou seja, aqueles com maior similaridade de conteúdo.

\subsection*{Classificação e Visualização das Recomendações}
Após calcular as similaridades entre o anime escolhido e os demais do dataset, o sistema classifica e exibe os resultados de acordo com o nível de similaridade:
\begin{itemize}
        \item \textbf{Alta} (\texttt{$\cos(\theta)$} $\geq$ 0.1): representada por três círculos verdes ($\bullet\;\bullet\;\bullet$) e uma barra totalmente preenchida, indicando forte similaridade (acima de 10\%).
        \item \textbf{Média} (0.08 $\leq$ \texttt{$\cos(\theta)$} $<$ 0.1): representada por dois círculos amarelos e um vazio ($\bullet\;\bullet\;\circ$) e uma barra parcialmente preenchida, indicando similaridade intermediária (entre 8\% e 10\%).
        \item \textbf{Baixa} (\texttt{$\cos(\theta)$} $<$ 0.08): representada por um círculo vermelho e dois vazios ($\bullet\;\circ\;\circ$) e uma barra pouco preenchida, indicando baixa similaridade (abaixo de 8\%).
\end{itemize}
Além disso, para cada recomendação, o sistema exibe:
\begin{itemize}
    \item O ranking da recomendação (\#1, \#2, ...).
    \item O título do anime recomendado.
    \item O nível de similaridade (alta, média ou baixa), o valor percentual e uma barra visual.
    \item Um trecho da sinopse do anime recomendado para facilitar a avaliação pelo usuário.
\end{itemize}
Essa visualização torna a experiência mais intuitiva, permitindo ao usuário entender rapidamente o grau de similaridade entre os animes recomendados e o anime escolhido.
\section{Estrutura do Projeto}

\subsection*{Descrição do Dataset}
O dataset utilizado pelo sistema está localizado em \texttt{src/constants/dataset.csv} e contém informações detalhadas sobre diversos animes. Cada linha representa um anime e as principais colunas utilizadas pelo sistema são:
\begin{itemize}
    \item \textbf{anime\_id}: Identificador único do anime.
    \item \textbf{Name}: Nome principal do anime.
    \item \textbf{sypnopsis}: Sinopse/descrição do enredo.
\end{itemize}



\subsection*{Descrição dos Arquivos e Diretórios}
O projeto é composto pelos seguintes arquivos e diretórios principais:
\begin{itemize}
    \item \textbf{main.py}: Script principal para execução do sistema de recomendação.
    \item \textbf{pyproject.toml}: Arquivo de configuração e dependências do projeto Python.
    \item \textbf{docs/}: Pasta de documentação do projeto (contém este arquivo .tex e outros documentos relacionados).
    \item \textbf{src/}: Código-fonte principal do sistema, contendo:
    \begin{itemize}
        \item \textbf{api/}: Módulos para integração com APIs externas (AniList, AniZip).
        \item \textbf{constants/}: Arquivos de dados e utilitários.
        \item \textbf{searcher/}: Funções de busca e pré-processamento de texto.
        \item \textbf{transformer/}: Implementação dos algoritmos de transformação e similaridade (TF-IDF, etc).
        \item \textbf{utils/}: Utilitários diversos, como manipulação de caminhos e proxies.
    \end{itemize}
\end{itemize}
\section{Como rodar o projeto}
Para rodar o projeto e instalar as dependências, siga os passos abaixo:

\begin{enumerate}
    \item Certifique-se de ter o Python 3.10 ou superior instalado em seu sistema.
    \item Abra o terminal na raiz do projeto (onde está o arquivo \texttt{pyproject.toml}).
    \item Instale as dependências do projeto com o comando:
    \begin{verbatim}
    pip install .
    \end{verbatim}
    \item Execute o sistema de recomendação com:
    \begin{verbatim}
    python main.py
    \end{verbatim}
\end{enumerate}

Os resultados das recomendações são exibidos no terminal, com destaques coloridos e emojis para facilitar a visualização e tornar a experiência mais agradável.

\section{Conclusão}
O sistema de recomendação Kori foi desenvolvido para sugerir animes similares ao que o usuário escolher, utilizando técnicas de processamento de linguagem natural e álgebra linear para analisar descrições e encontrar títulos com maior similaridade. Ele é útil para quem deseja descobrir novos animes de acordo com seus interesses, de forma automatizada e personalizada.
\end{document}
